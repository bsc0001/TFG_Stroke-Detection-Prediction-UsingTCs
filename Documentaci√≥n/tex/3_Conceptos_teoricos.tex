\capitulo{3}{Conceptos teóricos}

Este proyecto aplica la Ingeniería Informática en el ámbito de la Biomedicina, colaborando, más concretamente, con el diagnóstico de imágenes en Neurología. Esto implica un proceso de investigación y aprendizaje necesarios para el entendimiento del entorno, lo que supone mayor comprensión del vocabulario y objetivos específicos, así como para el desarrollo del proyecto en sí.

En este caso, nos enfocamos en conceptos básicos bastante concretos, dentro de dos áreas de conocimiento:
\begin{itemize}
	\item Neurología
	\item Técnicas de Imagen Biomédica
\end{itemize}

En el primer caso, los conceptos teóricos neurológicos que nos interesan son los referentes al accidente (o ataque) cerebro-vascular (ACV), más frecuentemente denominado ictus, y el tejido isquémico que resulta del mismo.

Por otro lado, dentro de las Técnicas de Imagen Biomédica, nos centramos en las Tomografías (Axiales) Computarizadas, conocidas comúnmente como TC o TAC. Principalmente, debemos tener claro el concepto de TC craneal, ya que es el tipo de imagen para el que este proyecto ha sido diseñado. Sin embargo, no está de más conocer por encima algo sobre las Resonancias Magnéticas (MRI por sus siglas en inglés \textit{Magnetic Resonance Imaging}), ya que su comparativa con las Tomografías Computarizadas dan más sentido e importancia a este proyecto.


\section{Accidente Cerebrovascular o Ictus}




\section{Tomografía Computarizada (TC/TAC)}




\subsection{TC o TAC craneal}




\subsection{Alternativa a la Resonancia Magnética (MRI)}





\indexspace
\indexspace
\indexspace
\indexspace
Algunos conceptos teóricos de \LaTeX \footnote{Créditos a los proyectos de Álvaro López Cantero: Configurador de Presupuestos y Roberto Izquierdo Amo: PLQuiz}.


\subsubsection{Subsubsecciones}

Y sub - sub - secciones. 


\section{Listas de items}

Existen tres posibilidades:

\begin{itemize}
	\item primer item.
	\item segundo item.
\end{itemize}

\begin{enumerate}
	\item primer item.
	\item segundo item.
\end{enumerate}

\begin{description}
	\item[Primer item] más información sobre el primer item.
	\item[Segundo item] más información sobre el segundo item.
\end{description}
	
\begin{itemize}
\item 
\end{itemize}

\section{Tablas}

Igualmente se pueden usar los comandos específicos de \LaTeX o bien usar alguno de los comandos de la plantilla.

\tablaSmall{Herramientas y tecnologías utilizadas en cada parte del proyecto}{l c c c c}{herramientasportipodeuso}
{ \multicolumn{1}{l}{Herramientas} & App AngularJS & API REST & BD & Memoria \\}{ 
HTML5 & X & & &\\
CSS3 & X & & &\\
BOOTSTRAP & X & & &\\
JavaScript & X & & &\\
AngularJS & X & & &\\
Bower & X & & &\\
PHP & & X & &\\
Karma + Jasmine & X & & &\\
Slim framework & & X & &\\
Idiorm & & X & &\\
Composer & & X & &\\
JSON & X & X & &\\
PhpStorm & X & X & &\\
MySQL & & & X &\\
PhpMyAdmin & & & X &\\
Git + BitBucket & X & X & X & X\\
Mik\TeX{} & & & & X\\
\TeX{}Maker & & & & X\\
Astah & & & & X\\
Balsamiq Mockups & X & & &\\
VersionOne & X & X & X & X\\
} 

